\documentclass{article}

\usepackage[utf8]{inputenc}
\usepackage{graphicx}
\usepackage{caption}
\usepackage{subcaption}

\begin{document}
\section*{Introduction}
For the final project we were tasked to use what we have learned on a dataset of our choice. 
Since we both watch Youtube, we decided to work with a Youtube data set from the Kaggle collection. 
This dataset contains multiple sets of Youtube metadata from many countries around the world. 
Each country has the same columns but a wide variety of different Youtube videos and categories. 
In this project we focused on the USA data set since that had the most familiar videos. 
From the Youtube dataset our goal was to extract useful information about the trending videos; what videos are trending and how fast do they trend?
Using this information we wanted to use a decision tree to predict how long it will take for a video to go trending. 
There is a lot of information in these datasets that we are not using but would not aid the decision tree in its training or prediction.
Overall, it was interesting challenge for the decision tree algorithm to predict when a video will go trending.
\section*{Implementation}
%%TALK about exactly what our code is doing to achieve the goal
\section*{Results}
%%Why is the decision tree accuracy is so bad
%%What about our data makes this hard
    %%it is because we need to use the words and maybe have a Machine leanring algorithm
%% we then tried the same approach for each individual category
    %%but that was bad
%%finally we tried predicting the category because it is easier
\section*{Conclusion}
    %% There wasn't enough useful information in numbers to make a good ecision tree
    %% to have good results you would have to rely on parsing the testing data or do some statistical analysis on the numbers


\end{document}